%
% Assignment 0c for CS3530 Computational Theory:
% LaTeX
% Fall 2024
%
% Problems taken from Sipser
%

\documentclass{article}

\usepackage[margin=1in]{geometry}
\usepackage{amsfonts}
\usepackage{amsmath}
\usepackage[english]{babel}
\usepackage[utf8]{inputenc}
\usepackage{ae,aecompl}
\usepackage{emp,ifpdf}
\usepackage{graphicx}

\ifpdf\DeclareGraphicsRule{*}{mps}{*}{}\fi

\empprelude{input boxes; input theory}

% skip for paragraphs, don't indent
\addtolength{\parskip}{0.5\baselineskip}
\parindent=0pt

\begin{document}
\begin{empfile}

\begin{center}
\textbf{\Large CS 3530: Assignment 0c} \\[2mm]
Fall 2024

Hayden Hawley
\end{center}

\raggedright

\section*{Exercise 0.6abcde (10 points)}

\subsection*{Problem}

Let $X$ be the set $\{ 1, 2, 3, 4, 5 \}$ and $Y$ be the set $\{ 6, 7, 8, 9, 10 \}$.  The unary function
$f: X \rightarrow Y$ and the binary function $g: X \times Y \rightarrow Y$ are described in the following
tables.

\begin{center}
\begin{tabular}{cc}
    
\begin{tabular}{c|c}
  $n$ & $f(n)$ \\
  \hline
  1 & 6 \\
  2 & 7 \\
  3 & 6 \\
  4 & 7 \\
  5 & 6 \\
\end{tabular}

&

\begin{tabular}{c|ccccc}
  $g$ & 6 & 7 & 8 & 9 & 10 \\
  \hline
  1 & 10 & 10 & 10 & 10 & 10 \\
  2 & 7 & 8 & 9 & 10 & 6 \\
  3 & 7 & 7 & 8 & 8 & 9 \\
  4 & 9 & 8 & 7 & 6 & 10 \\
  5 & 6 & 6 & 6 & 6 & 6 \\
\end{tabular}
\\
\end{tabular}
\end{center}

\begin{itemize}
\item[a.] What is the value of $f(2)$?

\textbf{Solution}

7

\item[b.] What are the range and domain of $f$?

\textbf{Solution}

Domain is \{1,2,3,4,5\}.

Range is \{6,7\}.

\item[c.] What is the value of $g(2, 10)$?

\textbf{Solution}

6

\item[d.] What are the range and domain of $g$?

\textbf{Solution}

Domain is $\{1,2,3,4,5\} \times \{6,7,8,9,10\}.$

Range is \{6,7,8,9,10\}.

\item[e.] What is the value of $g(4, f(4))$?

\textbf{Solution}

8 (because f(4) is 7, so g(4,7))

\end{itemize}

\section*{Exercise 0.8 (5 points)}

\subsection*{Problem}

Consider the undirected graph $G = (V, E)$ where $V$, the set of nodes, is $\{ 1, 2, 3, 4 \}$ and $E$, the
set of edges, is $\{ \{ 1, 2 \}, \{ 2, 3 \}, \{ 1, 3 \}, \{ 2, 4 \}, \{ 1, 4 \}, \}$.  Draw the graph
$G$.

\textbf{Solution}

\begin{center}
\begin{emp}(0,0)
  bignodes;
  undirectededges;
  u := 1.5cm;
  node.a1("1"); a1.c=(0,0);
  node.a2("2"); a2.c=(2u,0);
  node.a3("3"); a3.c=(2u,-2u);
  node.a4("4"); a4.c=(0,-2u);
  drawboxed(a1,a2,a3,a4);
  sedge(a1,a2); % Edge {1, 2}
  sedge(a2,a3); % Edge {2, 3}
  sedge(a1,a3); % Edge {1, 3}
  sedge(a2,a4); % Edge {2, 4}
  sedge(a1,a4); % Edge {1, 4}
\end{emp}
\end{center}

\section*{Exercise 0.9 (5 points)}

\subsection*{Problem}

Write a formal description of the following graph.

\begin{center}
\begin{emp}(0,0)
  bignodes;
  undirectededges;
  u := 1.5cm;
  node.a1("1"); a1.c=(0,0);
  node.a2("2"); a2.c=(0,-u);
  node.a3("3"); a3.c=(0,-2u);
  node.a4("4"); a4.c=(2u,0);
  node.a5("5"); a5.c=(2u,-u);
  node.a6("6"); a6.c=(2u,-2u);
  drawboxed(a1,a2,a3,a4,a5,a6);
  sedge(a1,a4); sedge(a1,a5); sedge(a1,a6);
  sedge(a2,a4); sedge(a2,a5); sedge(a2,a6);
  sedge(a3,a4); sedge(a3,a5); sedge(a3,a6);
\end{emp}
\end{center}



\textbf{Solution}

An undirected graph where each node in the set \{1,2,3\} is connected to each node in the set \{4,5,6\}.

\end{empfile}
\immediate\write18{mpost -tex=latex \jobname}
\end{document}
