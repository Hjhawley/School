%
% Assignment 0b for CS3530 Computational Theory:
% LaTeX
% Fall 2024
%
% Problems taken from Sipser
%

\documentclass{article}

\usepackage[margin=1in]{geometry}
\usepackage{amsfonts}
\usepackage{amsmath}
\usepackage[english]{babel}
\usepackage[utf8]{inputenc}
\usepackage{ae,aecompl}
\usepackage{emp,ifpdf}
\usepackage{graphicx}

\ifpdf\DeclareGraphicsRule{*}{mps}{*}{}\fi

\empprelude{input boxes; input theory}

% skip for paragraphs, don't indent
\addtolength{\parskip}{0.5\baselineskip}
\parindent=0pt

\begin{document}
\begin{empfile}

\begin{center}
\textbf{\Large CS 3530: Assignment 0b} \\[2mm]
Fall 2024

\emph{Hayden Hawley}
\end{center}

\raggedright

\section*{Exercise 0.1def (6 points)}

\subsection*{Problem}

Examine the following formal descriptions of sets so that you understand which members they contain.  Write a short informal English description of each set.

\begin{itemize}
\item[d.] $\{ n | n = 2m $ for some $m$ in $\mathbb{N}$, and $n = 3k$ for some $k$ in $\mathbb{N} \}$

\textbf{Solution}

The set of all numbers n where n is a multiple of both 2 and 3.

\item[e.] $\{ w | w $ is a string of $0$s and $1$s and $w$ equals the reverse of $w \}$

\textbf{Solution}

The set of all numbers w where w is a palindrome made of 1's and 0's.

\item[f.] $\{ n | n $ is an integer and $n = n + 1 \}$

\textbf{Solution}

The set of all numbers n where n is 1 greater than itself (impossible; this is an empty set)

\end{itemize}

\section*{Exercise 0.2b (2 points)}

\subsection*{Problem}

Write formal descriptions of the following sets.

\begin{itemize}
\item[b.] The set containing all integers that are greater than 5.

\textbf{Solution}

\{n \mid n \in \mathbb{Z}, \, n > 5\}

\end{itemize}

\section*{Exercise 0.3abcdef (12 points)}

\subsection*{Problem}

Let $A$ be the set $\{ x, y, z \}$ and $B$ be the set $\{ x, y \}$.

\begin{itemize}
\item[a.] Is $A$ a subset of $B$?

\textbf{Solution}

No, z is not part of B

\item[b.] Is $B$ a subset of $A$?

\textbf{Solution}

Yes

\item[c.] What is $A \cup B$?

\textbf{Solution}

$A \cup B$ = \{x, y, z\}

\item[d.] What is $A \cap B$?

\textbf{Solution}

$A \cap B$ = \{x, y\}

\item[e.] What is $A \times B$?

\textbf{Solution}

$A \times B$ = \{(x, x), (x, y), (y, x), (y, y), (z, x), (z, y)\}

\item[f.] What is the power set of $B$?

\textbf{Solution}

The set of all subsets of $B$, which is: \{\emptyset, \{x\}, \{y\}, \{x, y\}\}.

\end{itemize}

\end{empfile}
\immediate\write18{mpost -tex=latex \jobname}
\end{document}
